\documentclass[16pt,openbib]{beamer}

\author{Pritvi Jheengut @zcoldplayer} %\thanks{\and}}
\title{A little stroll from Floss to Hacktoberfest}

\subtitle{Importance of Licensing \\
Hacking and Free Software \\
Git!!!!!!!!  GitHub\\
Hacktoberfest\\
Digital Ocean\\
}

\subject{}

\usepackage[size=custom,width=64,height=36,orientation=landscape,
  scale=1.75]{beamerposter}

\usepackage{lmodern}
\usepackage[T1]{fontenc}
\usepackage[utf8]{inputenc}

% Eliminate errors such as
% \LaTeX\ Font Warning: Font shape `T1/cmss/m/n' in size <4> not available
% \LaTeX\ Font Warning: Size substitutions with differences up to 1.0pt

\usepackage{hyperref}
\hypersetup{
  pdfcreator=pdflatex,
  pdffitwindow=true
}

\usepackage{verbatim}

\usepackage[french,UKenglish]{babel}
\usepackage{listings}

% \usepackage{pgf}
% \usepackage{xy}[all]
\usepackage{graphicx}

\usepackage{tikz}
\usetikzlibrary{shapes.geometric, arrows,positioning}

\usepackage{smartdiagram}

% \usepackage{booktabs}

\mode<presentation>

  \useoutertheme{infolines}

  \definecolor{light}{HTML}{121B2F}
  \definecolor{confblue}{HTML}{5D8EAF}
  \definecolor{confred}{HTML}{7B1F7C}
  \definecolor{confyellow}{HTML}{121B2F}
  \definecolor{conftext}{HTML}{77AABB}

  \setbeamercolor{alerted text}
  {bg=confblue,fg=conftext}

  \setbeamercolor{block title}
  {bg=confred,fg=conftext}

  \setbeamercolor{block body}
  {bg=confblue,fg=-confyellow}

  \setbeamercolor{frametitle}
  {parent=confred,fg=-confyellow}

  \setbeamercolor{normal text}
  {bg=confred,fg=confblue}

  \setbeamercolor*{palette primary}
  {use=structure,fg=confblue,bg=structure.fg}

  \setbeamercolor*{palette secondary}
  {use=structure,fg=-confblue,bg=structure.fg!85!confblue}

  \setbeamercolor*{palette tertiary}
  {use=structure,fg=-confblue,bg=structure.fg!70!confblue}

  \setbeamercolor{structure}
  {bg=confyellow,fg=confblue}

  \setbeamercolor{Title bar}
  {fg=confred!90}

  \setbeamercolor{titlelike}
  {parent=confyellow,bg=structure.fg!30!-confyellow,fg=black!85}

  \setbeamercovered{transparent,dynamic}

  \setbeamersize{text margin left=8cm,text margin right=2cm}

  \setbeamertemplate{blocks}[rounded][shadow=true]

  \setbeamertemplate{background canvas}[vertical shading]
  [top=confred,middle=confyellow,bottom=confyellow]


% \mode<handout>{\beamertemplatesolidbackgroundcolor{black!50}}

\mode<all>

\begin{document}

\date[Hacktoberfest BootCamp]{Hacktoberfest BootCamp - 21 September 2019}

\frame{
  \frametitle{}

  \huge
  \maketitle

}

\section{Intellectual Property Alert : Overview of the Copyleft
  Licenses}

\frame{
  \frametitle{Copyleft License Attribution}

  Made with love using beamer, \LaTeX\ and git.\\
  You can view at \href{}{Hacktoberfest 2019}

  \begin{alertblock}{This work is licensed under the \LaTeX\ Project
      Public License.}
    To view a copy of this license, visit \\
    \url{https://www.latex-project.org/lppl.txt}
  \end{alertblock}

  \begin{alertblock}{This work is licensed under the Creative
      Commons Attribution-ShareAlike 4.0 International License.}
    To view a copy of this license, visit \\
    \href{http://creativecommons.org/licenses/by-sa/4.0/}{CC BY-SA}
    or \\

    send a letter to \\
    Creative Commons, \\
    PO Box 1866, \\
    Mountain View, \\
    CA 94042, \\
    USA. \\
  \end{alertblock}

}

\section{Greetings}
\subsection{About the Author}

\frame{
  \frametitle{Who Am I}

  \large

  \href{http://slackware.com/}{Geek@Slackware}

  \href{https://twitter.com/zcoldplayer}{twitter @zcoldplayer}

  \href{https://xmail.net/z.coldplayer}{zcoldplayer xmail Website}

  \href{http://metservice.intnet.mu/}{Work
    : SMTT@Meteorological.Services.mu}

  Active in many User group LUGM, MMC, MSCC, FECM, GDG\_MU \\
  and several other hackathons

  Passionate about how and why things work.

  Fervour Advocate of Free Libre and Open Source Software.

}

\subsection{About the Audience}

\frame{
  \frametitle{Who are you}

  \large

  \begin{block}{Would you mind tell me who you are?}
    Some hints:
  \end{block}

  \begin{enumerate}
  \item @twitter\_handle
  \item where you work
  \item email you want to share
  \item Hobbies
  \item purpose and expectations of this session
  \end{enumerate}

}

\section{Community Groups in Mauritius}

\frame{
  \frametitle{Community Groups in Mauritius}

  \begin{block}{Healthy growth of Community Groups in Mauritius}
    This turn of the century has seen an uprising of Community groups
    in Mauritius in the field of the Digital World. The diversity has
    helped the exchange and sharing of innovative ideas, experience,
    bleeding edge technology, upcoming events, conferences in the
    Digital Island of the Republic of Mauritius.
  \end{block}

  This is a list of some of the active communities in Digital
  Mauritius.

  \begin{itemize}
  \item Linux User Group Meta, LUGM
  \item Mauritius Software Craftsmanship Community, MSCC
  \item Mauritius Makers Community, MMC
  \item Front-End Coders Mauritius, FECM
  \item PHP User Group of Mauritius, phpMauritiusUG
  \item Symfonymu
  \item Google Developers Group Mauritius, GDG\_M
  \item Digital Marketing Mauritius
  \end{itemize}

}

\frame{
  \frametitle{Dennis Ritchie}
 
 \large 
 
 Dennis MacAlistair Ritchie was an American computer scientist.
 
 He created the C programming language and, with long-time 
 colleague Ken Thompson,
 
 the Unix operating system.
 
 The 12th of october 2019 marks his 8th death anniversary. 
 
 A little tribute to him.
  }
  
\frame{
  \frametitle{Questions}

\huge Questions !!  
  }
  

\frame{
  \frametitle{About YOUR experience}

  \Huge
  
  Have you ever participated in \#Hacktoberfest

}  
  
\frame{
  \frametitle{Do you License your creations?}

  \Huge Do you License your creations ?
  
  What type of license do you usually use ?
  
  }
  

\frame{
  \frametitle{Hacking and Free Software }
  
  \Huge Hacking and Free Software have always worked together since
  the Dawn of computing
  
  Do you Hack ?
  }
  

\frame{
  \frametitle{Hacking and Free Software }
  
  \Huge 
  
  To hack means to be very good at something but the work to be done
  is not necessarily a quality work.
  
  }
  
\frame{
  \frametitle{Hacking and Free Software }
  
  \Huge 
  
  In essence, hacking is adding features to something which the 
  original author never thought about.
  
  }
  
\frame{
  \frametitle{Free Software }
  
  \Huge  Free Software has more emphasis upon end users
  
  Any change done to Free Software has to be published.

  }
  
\frame{
  \frametitle{Open Source Software }

  \Huge Open Source Software  has more emphasis upon developers
  }
  
\frame{
  \frametitle{Linux and Linus Torvalds}
  
  \Huge
  
  What is Linux ??
  
  Linus Torvalds has released Linux under the GPL.  
  
  }
  
\frame{
  \frametitle{Problems within the Linux Kernel}

  \Huge Huge project with millions of lines of codes,
  
  Several thousands of developers send an increasing number
  of patches per day
  }
  
\frame{
  \frametitle{Bitkeeper}

  \Huge Initially, the patches were managed through email
  
  Later, Bitkeeper, a closed source versioning content system 
  was used to manage the Linux Kernel patches.
  
  }
  
\frame{
  \frametitle{Git!!!!!!!!! }

  \Huge Being closed sourced became impracticable, a solution 
  was needed. 
  
  In a few days, Linus Torvalds wrote git using C primarily.
  
  }
  
\frame{
  \frametitle{GitHub - part 1}
  
  \huge
  
  \href{https://github.com}
  {GitHub provides hosting for software development version control
  using Git.}
  
  what are your favourite Free and Open Source project??
  
  }
  
\frame{
  \frametitle{GitHub - part 2}
  
  \huge

  Last year Github was acquired by and is now subsidiary a Microsoft
  while offering all management functionality of Git as well as
  adding its own features but they are neither Free nor Open Source
  software but it is free of cost to use to a certain limit.
  
  }
  
\frame{
  \frametitle{GitHub - part 3}
  
  \huge
  
  GitHub provides access control and several collaboration features 
  such as bug tracking, feature requests, task management, and 
  wikis for every project.
  
  }
  
\frame{
  \frametitle{GitHub -- Be social}
  
  \Huge
  
  \href{https://help.github.com/en/articles/be-social}
  {You can interact with people, repositories, and organizations on
  GitHub. See what others are working on and who they're connecting 
  with from your personal dashboard.}
  
  }
  
\frame{
  \frametitle{Digital Ocean}
  
  \Huge
  
  \href{https://www.digitalocean.com/}
  {Digital Ocean is a cloud computing service provider so developers
  and their teams use to build software.}
  
  }
  
\frame{
  \frametitle{what is Hacktoberfest}
  
  \Huge
  
  \href{https://hacktoberfest.digitalocean.com/}
  {Hacktoberfest is an annual month-long event held in the month of
  October organised by GitHub and Digital Ocean.}
  
  }
  
\frame{
  \frametitle{Why Hacktoberfest}
  
  \Huge
  
  A marathon of coding, hacking festival is called a Hackathon.
  
  A month long of beer festival, Volksfest, is called Oktoberfest.
  
  }
  
\frame{
  \frametitle{Hacktoberfest for coders, developers, hackers}
  
  \Huge
  This is a month to celebrate and encourage developers to get
  involved and contribute to open source projects.
  
  }
  
\frame{
  \frametitle{How to Hacktoberfest}
    
  \Huge
  
  Since this festival is about Free and Open Source Software.
  
  Hacktoberfest is open to everyone in our global community.
  
  }
  
\frame{
  \frametitle{How to Hacktoberfest}
    
  \Huge
  
  The next step is to submit four quality pull requests to public
  GitHub repositories.
  
  }
  
\frame{
  \frametitle{How to Hacktoberfest}
    
  \Huge
  
  \href{https://hacktoberfest.digitalocean.com/}
  {Register at hacktoberfest.digitalocean.com anytime between
  October 1 and October 31.}
  
  }
  
\frame{
  \frametitle{How to Hacktoberfest}
    
  \Huge
  
  Work on issues with the label Hacktoberfest on GitHub projects and
  when finished send the pull request.
  
  }
  
\frame{
  \frametitle{Hacktoberfest Goodies}
    
  \Huge
  
  Upon completion of four pull requests, you will receive a 
  Hacktoberfest t-shirt and stickers by November.
  
  }
  
  
    
\frame{
  \frametitle{Questions}
  \Huge Questions !!  
  }
  
\end{document}
